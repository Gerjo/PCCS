\documentclass[acmtocl,acmnow]{acmtrans2m}

\newtheorem{theorem}{Theorem}[section]
\newtheorem{conjecture}[theorem]{Conjecture}
\newtheorem{corollary}[theorem]{Corollary}
\newtheorem{proposition}[theorem]{Proposition}
\newtheorem{lemma}[theorem]{Lemma}
\newdef{definition}[theorem]{Definition}
\newdef{remark}[theorem]{Remark}
\markboth{Game Engine Development}{CpPhantom}
\title{C++ Phantom}
\author{Sander Brattinga\newline Nico Glas\newline Gerard Meier}
\begin{abstract} 
This article is intended for game engine creators. It will feature the development process. Why we did some nasty things and more.\newline
Some say they are geniuses, while others say they are just lunatics... all we know is they created an engine. (Jeremy Clarkson voice)
\end{abstract}

\category{Engine Development}{}{Game engine in C++ that simulates the HTML5 canvas.}

\terms{} 

\keywords{}
\begin{document}

\begin{bottomstuff}
\end{bottomstuff}

\maketitle
\section{Blastoff}
We actually first started thinking of a game. We developed the engine and game simultaneously so we knew instantly what we needed from the engine. The engine we created is component based and sort of like the Phantom flash game engine. 

\subsection{Design}
We wanted to create an engine that is HTML5 syntax compatible. This means we created a single context, canvas or what ever you want to call it. You can draw things with the same commands you expect to work in HTML5. This means the functions beginPath, rect, arc, lineTo and more are available to use.\newline
Another important point is that we wanted the engine to be completely component based. This makes it easy to create a nice hierarchy with almost unlimited possibilities.

\subsection{External dependencies}
For rendering we decided to use OpenGL. OpenGL is cross platform and commonly used for the rendering of graphics. For window creation, audio and input we decided on using SDL. During development we decided to change it to FreeGLUT. The reason for this was that FreeGLUT had less dependencies. Afterwards it was a very bad choice. SDL has a lot more useful features for engine development.

\section{The engine}
\subsection{Component architecture}
hints: iterative issues, why, solve, tree
\subsection{Graphics}
hints: VBOs, texure loading (png), shapes
\subsection{Audio}
hints: OpenAL
\subsection{Input}
hints: FreeGLUT ftl
\subsection{Artificial Intelligence}
hints: Pathfinding
\subsection{Physics}
hints: Vector, InertiaMover, Pulses

\begin{thebibliography}{}

\end{thebibliography}

\begin{received}
\end{received}
\end{document}

